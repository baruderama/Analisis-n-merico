\documentclass[]{article}
\usepackage{lmodern}
\usepackage{amssymb,amsmath}
\usepackage{ifxetex,ifluatex}
\usepackage{fixltx2e} % provides \textsubscript
\ifnum 0\ifxetex 1\fi\ifluatex 1\fi=0 % if pdftex
  \usepackage[T1]{fontenc}
  \usepackage[utf8]{inputenc}
\else % if luatex or xelatex
  \ifxetex
    \usepackage{mathspec}
  \else
    \usepackage{fontspec}
  \fi
  \defaultfontfeatures{Ligatures=TeX,Scale=MatchLowercase}
\fi
% use upquote if available, for straight quotes in verbatim environments
\IfFileExists{upquote.sty}{\usepackage{upquote}}{}
% use microtype if available
\IfFileExists{microtype.sty}{%
\usepackage{microtype}
\UseMicrotypeSet[protrusion]{basicmath} % disable protrusion for tt fonts
}{}
\usepackage[margin=1in]{geometry}
\usepackage{hyperref}
\hypersetup{unicode=true,
            pdfborder={0 0 0},
            breaklinks=true}
\urlstyle{same}  % don't use monospace font for urls
\usepackage{color}
\usepackage{fancyvrb}
\newcommand{\VerbBar}{|}
\newcommand{\VERB}{\Verb[commandchars=\\\{\}]}
\DefineVerbatimEnvironment{Highlighting}{Verbatim}{commandchars=\\\{\}}
% Add ',fontsize=\small' for more characters per line
\usepackage{framed}
\definecolor{shadecolor}{RGB}{248,248,248}
\newenvironment{Shaded}{\begin{snugshade}}{\end{snugshade}}
\newcommand{\AlertTok}[1]{\textcolor[rgb]{0.94,0.16,0.16}{#1}}
\newcommand{\AnnotationTok}[1]{\textcolor[rgb]{0.56,0.35,0.01}{\textbf{\textit{#1}}}}
\newcommand{\AttributeTok}[1]{\textcolor[rgb]{0.77,0.63,0.00}{#1}}
\newcommand{\BaseNTok}[1]{\textcolor[rgb]{0.00,0.00,0.81}{#1}}
\newcommand{\BuiltInTok}[1]{#1}
\newcommand{\CharTok}[1]{\textcolor[rgb]{0.31,0.60,0.02}{#1}}
\newcommand{\CommentTok}[1]{\textcolor[rgb]{0.56,0.35,0.01}{\textit{#1}}}
\newcommand{\CommentVarTok}[1]{\textcolor[rgb]{0.56,0.35,0.01}{\textbf{\textit{#1}}}}
\newcommand{\ConstantTok}[1]{\textcolor[rgb]{0.00,0.00,0.00}{#1}}
\newcommand{\ControlFlowTok}[1]{\textcolor[rgb]{0.13,0.29,0.53}{\textbf{#1}}}
\newcommand{\DataTypeTok}[1]{\textcolor[rgb]{0.13,0.29,0.53}{#1}}
\newcommand{\DecValTok}[1]{\textcolor[rgb]{0.00,0.00,0.81}{#1}}
\newcommand{\DocumentationTok}[1]{\textcolor[rgb]{0.56,0.35,0.01}{\textbf{\textit{#1}}}}
\newcommand{\ErrorTok}[1]{\textcolor[rgb]{0.64,0.00,0.00}{\textbf{#1}}}
\newcommand{\ExtensionTok}[1]{#1}
\newcommand{\FloatTok}[1]{\textcolor[rgb]{0.00,0.00,0.81}{#1}}
\newcommand{\FunctionTok}[1]{\textcolor[rgb]{0.00,0.00,0.00}{#1}}
\newcommand{\ImportTok}[1]{#1}
\newcommand{\InformationTok}[1]{\textcolor[rgb]{0.56,0.35,0.01}{\textbf{\textit{#1}}}}
\newcommand{\KeywordTok}[1]{\textcolor[rgb]{0.13,0.29,0.53}{\textbf{#1}}}
\newcommand{\NormalTok}[1]{#1}
\newcommand{\OperatorTok}[1]{\textcolor[rgb]{0.81,0.36,0.00}{\textbf{#1}}}
\newcommand{\OtherTok}[1]{\textcolor[rgb]{0.56,0.35,0.01}{#1}}
\newcommand{\PreprocessorTok}[1]{\textcolor[rgb]{0.56,0.35,0.01}{\textit{#1}}}
\newcommand{\RegionMarkerTok}[1]{#1}
\newcommand{\SpecialCharTok}[1]{\textcolor[rgb]{0.00,0.00,0.00}{#1}}
\newcommand{\SpecialStringTok}[1]{\textcolor[rgb]{0.31,0.60,0.02}{#1}}
\newcommand{\StringTok}[1]{\textcolor[rgb]{0.31,0.60,0.02}{#1}}
\newcommand{\VariableTok}[1]{\textcolor[rgb]{0.00,0.00,0.00}{#1}}
\newcommand{\VerbatimStringTok}[1]{\textcolor[rgb]{0.31,0.60,0.02}{#1}}
\newcommand{\WarningTok}[1]{\textcolor[rgb]{0.56,0.35,0.01}{\textbf{\textit{#1}}}}
\usepackage{longtable,booktabs}
\usepackage{graphicx,grffile}
\makeatletter
\def\maxwidth{\ifdim\Gin@nat@width>\linewidth\linewidth\else\Gin@nat@width\fi}
\def\maxheight{\ifdim\Gin@nat@height>\textheight\textheight\else\Gin@nat@height\fi}
\makeatother
% Scale images if necessary, so that they will not overflow the page
% margins by default, and it is still possible to overwrite the defaults
% using explicit options in \includegraphics[width, height, ...]{}
\setkeys{Gin}{width=\maxwidth,height=\maxheight,keepaspectratio}
\IfFileExists{parskip.sty}{%
\usepackage{parskip}
}{% else
\setlength{\parindent}{0pt}
\setlength{\parskip}{6pt plus 2pt minus 1pt}
}
\setlength{\emergencystretch}{3em}  % prevent overfull lines
\providecommand{\tightlist}{%
  \setlength{\itemsep}{0pt}\setlength{\parskip}{0pt}}
\setcounter{secnumdepth}{0}
% Redefines (sub)paragraphs to behave more like sections
\ifx\paragraph\undefined\else
\let\oldparagraph\paragraph
\renewcommand{\paragraph}[1]{\oldparagraph{#1}\mbox{}}
\fi
\ifx\subparagraph\undefined\else
\let\oldsubparagraph\subparagraph
\renewcommand{\subparagraph}[1]{\oldsubparagraph{#1}\mbox{}}
\fi

%%% Use protect on footnotes to avoid problems with footnotes in titles
\let\rmarkdownfootnote\footnote%
\def\footnote{\protect\rmarkdownfootnote}

%%% Change title format to be more compact
\usepackage{titling}

% Create subtitle command for use in maketitle
\providecommand{\subtitle}[1]{
  \posttitle{
    \begin{center}\large#1\end{center}
    }
}

\setlength{\droptitle}{-2em}

  \title{}
    \pretitle{\vspace{\droptitle}}
  \posttitle{}
    \author{}
    \preauthor{}\postauthor{}
    \date{}
    \predate{}\postdate{}
  

\begin{document}

\begin{longtable}[]{@{}l@{}}
\toprule
\endhead
title: ``R Notebook''\tabularnewline
output: html\_notebook\tabularnewline
\bottomrule
\end{longtable}

Biseccion e\^{}x - pi*x

\begin{Shaded}
\begin{Highlighting}[]
\KeywordTok{rm}\NormalTok{(}\DataTypeTok{list=}\KeywordTok{ls}\NormalTok{())}
\NormalTok{Fx =}\StringTok{ }\ControlFlowTok{function}\NormalTok{(x)\{}
  \KeywordTok{return}\NormalTok{ ((}\KeywordTok{exp}\NormalTok{(x)) }\OperatorTok{-}\StringTok{ }\NormalTok{(pi}\OperatorTok{*}\NormalTok{x))}
\NormalTok{\}}

\NormalTok{biseccion =}\StringTok{ }\ControlFlowTok{function}\NormalTok{(a, b, error)\{}
  \CommentTok{#Para graficar se crea una secuencia de números entre el rango [a,b]}
\NormalTok{  x =}\StringTok{ }\KeywordTok{seq}\NormalTok{(a, b, }\FloatTok{0.1}\NormalTok{)}
  \KeywordTok{plot}\NormalTok{(x, }\KeywordTok{Fx}\NormalTok{(x), }\DataTypeTok{type =} \StringTok{"l"}\NormalTok{, }\DataTypeTok{col =} \StringTok{"red"}\NormalTok{)}
  \KeywordTok{abline}\NormalTok{(}\DataTypeTok{h =} \DecValTok{0}\NormalTok{, }\DataTypeTok{v =} \DecValTok{0}\NormalTok{, }\DataTypeTok{col =} \StringTok{"blue"}\NormalTok{)}
\NormalTok{  xr =}\StringTok{ }\DecValTok{0}
\NormalTok{  fxr =}\StringTok{ }\DecValTok{0}
\NormalTok{  contador =}\StringTok{ }\DecValTok{0}
  \ControlFlowTok{if}\NormalTok{ (}\KeywordTok{Fx}\NormalTok{(a) }\OperatorTok{*}\StringTok{ }\KeywordTok{Fx}\NormalTok{(b) }\OperatorTok{>}\StringTok{ }\DecValTok{0}\NormalTok{)\{}
    \KeywordTok{cat}\NormalTok{(}\StringTok{"No se puede aplicar el método"}\NormalTok{)}
\NormalTok{  \}}
  \ControlFlowTok{else}\NormalTok{ \{}
\NormalTok{    xr =}\StringTok{ }\NormalTok{(a }\OperatorTok{+}\StringTok{ }\NormalTok{b) }\OperatorTok{/}\StringTok{ }\FloatTok{2.0}
\NormalTok{    fxr =}\StringTok{ }\KeywordTok{Fx}\NormalTok{(xr)}
    \ControlFlowTok{if}\NormalTok{ (fxr }\OperatorTok{<=}\StringTok{ }\NormalTok{error)\{}
      \ControlFlowTok{break}
\NormalTok{    \} }\ControlFlowTok{else}\NormalTok{ \{}
      \ControlFlowTok{if}\NormalTok{ (fxr }\OperatorTok{*}\StringTok{ }\KeywordTok{Fx}\NormalTok{(a) }\OperatorTok{>}\StringTok{ }\DecValTok{0}\NormalTok{)}
\NormalTok{        a =}\StringTok{ }\NormalTok{xr}
      \ControlFlowTok{if}\NormalTok{ (fxr }\OperatorTok{*}\StringTok{ }\KeywordTok{Fx}\NormalTok{(b) }\OperatorTok{>}\StringTok{ }\DecValTok{0}\NormalTok{)}
\NormalTok{        b =}\StringTok{ }\NormalTok{xr}
\NormalTok{    \}}
\NormalTok{    contador =}\StringTok{ }\NormalTok{contador }\OperatorTok{+}\StringTok{ }\DecValTok{1}
\NormalTok{  \}}
  \KeywordTok{cat}\NormalTok{(}\StringTok{"Iteracciones: "}\NormalTok{, contador, }\StringTok{"Resultado: "}\NormalTok{, xr, }\StringTok{"}\CharTok{\textbackslash{}n}\StringTok{"}\NormalTok{)}
\NormalTok{\}}

\KeywordTok{biseccion}\NormalTok{(}\DecValTok{0}\NormalTok{, }\DecValTok{1}\NormalTok{, }\FloatTok{10e-8}\NormalTok{)}
\end{Highlighting}
\end{Shaded}

\includegraphics{Todos_files/figure-latex/unnamed-chunk-1-1.pdf}

\begin{verbatim}
## Iteracciones:  1 Resultado:  0.5
\end{verbatim}

Método de Newton e\^{}x - pi * x Derivada

\begin{Shaded}
\begin{Highlighting}[]
\KeywordTok{rm}\NormalTok{(}\DataTypeTok{list=}\KeywordTok{ls}\NormalTok{())}
\NormalTok{Fx =}\StringTok{ }\ControlFlowTok{function}\NormalTok{(x) }\KeywordTok{exp}\NormalTok{(x) }\OperatorTok{-}\StringTok{ }\NormalTok{pi }\OperatorTok{*}\StringTok{ }\NormalTok{x}
\NormalTok{Fx1 =}\StringTok{ }\ControlFlowTok{function}\NormalTok{(x) }\KeywordTok{exp}\NormalTok{(x) }\OperatorTok{-}\StringTok{ }\NormalTok{pi}

\NormalTok{Newton =}\StringTok{ }\ControlFlowTok{function}\NormalTok{(a, b, error)\{}
\NormalTok{  x =}\StringTok{ }\KeywordTok{seq}\NormalTok{(a, b, }\FloatTok{0.1}\NormalTok{)}
  \KeywordTok{plot}\NormalTok{(x, }\KeywordTok{Fx}\NormalTok{(x), }\DataTypeTok{type =} \StringTok{"l"}\NormalTok{, }\DataTypeTok{col =} \StringTok{"red"}\NormalTok{)}
  \KeywordTok{abline}\NormalTok{(}\DataTypeTok{h =} \DecValTok{0}\NormalTok{, }\DataTypeTok{v =} \DecValTok{0}\NormalTok{, }\DataTypeTok{col =} \StringTok{"blue"}\NormalTok{)}
  
\NormalTok{  x_}\DecValTok{0}\NormalTok{ =}\StringTok{ }\NormalTok{(a }\OperatorTok{+}\StringTok{ }\NormalTok{b) }\OperatorTok{/}\StringTok{ }\DecValTok{2}
  
\NormalTok{  contador =}\StringTok{ }\DecValTok{0}
\NormalTok{  dx =}\StringTok{ }\DecValTok{0}
  \ControlFlowTok{repeat}\NormalTok{ \{}
\NormalTok{    corr =}\StringTok{ }\KeywordTok{Fx}\NormalTok{(x_}\DecValTok{0}\NormalTok{) }\OperatorTok{/}\StringTok{ }\KeywordTok{Fx1}\NormalTok{(x_}\DecValTok{0}\NormalTok{)}
\NormalTok{    x_}\DecValTok{1}\NormalTok{ =}\StringTok{ }\NormalTok{x_}\DecValTok{0} \OperatorTok{-}\StringTok{ }\NormalTok{corr}
\NormalTok{    dx =}\StringTok{ }\KeywordTok{abs}\NormalTok{(corr)}
\NormalTok{    x_}\DecValTok{0}\NormalTok{ =}\StringTok{ }\NormalTok{x_}\DecValTok{1}
\NormalTok{    contador =}\StringTok{ }\NormalTok{contador }\OperatorTok{+}\StringTok{ }\DecValTok{1}
    
    \KeywordTok{cat}\NormalTok{(contador, dx, }\StringTok{"}\CharTok{\textbackslash{}n}\StringTok{"}\NormalTok{)}
    \ControlFlowTok{if}\NormalTok{ (dx }\OperatorTok{<=}\StringTok{ }\NormalTok{error)}
      \ControlFlowTok{break}
\NormalTok{  \}}
  \KeywordTok{cat}\NormalTok{ (}\StringTok{"Iteracciones: "}\NormalTok{, contador, }\StringTok{"Resultado: "}\NormalTok{, x_}\DecValTok{1}\NormalTok{, }\StringTok{"}\CharTok{\textbackslash{}n}\StringTok{"}\NormalTok{)}
\NormalTok{\}}
\KeywordTok{Newton}\NormalTok{(}\OperatorTok{-}\DecValTok{1}\NormalTok{, }\DecValTok{1}\NormalTok{, }\FloatTok{10e-8}\NormalTok{)}
\end{Highlighting}
\end{Shaded}

\includegraphics{Todos_files/figure-latex/unnamed-chunk-2-1.pdf}

\begin{verbatim}
## 1 0.4669422 
## 2 0.08287642 
## 3 0.003998516 
## 4 9.896266e-06 
## 5 6.078286e-11 
## Iteracciones:  5 Resultado:  0.553827
\end{verbatim}

Método del Punto Fijo

\begin{Shaded}
\begin{Highlighting}[]
\KeywordTok{rm}\NormalTok{(}\DataTypeTok{list=}\KeywordTok{ls}\NormalTok{())}
\NormalTok{Fx =}\StringTok{ }\ControlFlowTok{function}\NormalTok{(x) }\KeywordTok{exp}\NormalTok{(x) }\OperatorTok{/}\StringTok{ }\NormalTok{pi}
\NormalTok{Fx1 =}\StringTok{ }\ControlFlowTok{function}\NormalTok{(x) }\KeywordTok{log}\NormalTok{(x}\OperatorTok{*}\NormalTok{pi)}

\NormalTok{PuntoFijo =}\StringTok{ }\ControlFlowTok{function}\NormalTok{(a, b, error)\{}
\NormalTok{  xInicial =}\StringTok{ }\NormalTok{a}
\NormalTok{  x =}\StringTok{ }\KeywordTok{seq}\NormalTok{(a, b, }\FloatTok{0.1}\NormalTok{)}
  \KeywordTok{plot}\NormalTok{(x, }\KeywordTok{Fx}\NormalTok{(x), }\DataTypeTok{type =} \StringTok{"l"}\NormalTok{, }\DataTypeTok{col =} \StringTok{"orange"}\NormalTok{)}
  \KeywordTok{plot}\NormalTok{(x, }\KeywordTok{Fx1}\NormalTok{(x), }\DataTypeTok{type =} \StringTok{"l"}\NormalTok{, }\DataTypeTok{col =} \StringTok{"blue"}\NormalTok{)}
  \KeywordTok{abline}\NormalTok{(}\DataTypeTok{h =} \DecValTok{0}\NormalTok{, }\DataTypeTok{v =} \DecValTok{0}\NormalTok{, }\DataTypeTok{col =} \StringTok{"red"}\NormalTok{)}
  \ControlFlowTok{if}\NormalTok{ (}\KeywordTok{Fx}\NormalTok{(a) }\OperatorTok{<}\StringTok{ }\NormalTok{a }\OperatorTok{||}\StringTok{ }\KeywordTok{Fx}\NormalTok{(b) }\OperatorTok{<}\StringTok{ }\NormalTok{b)}
    \KeywordTok{cat}\NormalTok{(}\StringTok{"El intervalo no es valido}\CharTok{\textbackslash{}n}\StringTok{"}\NormalTok{)}
  \ControlFlowTok{else}\NormalTok{ \{}
\NormalTok{    x_}\DecValTok{0}\NormalTok{ =}\StringTok{ }\NormalTok{(a }\OperatorTok{+}\StringTok{ }\NormalTok{b) }\OperatorTok{/}\StringTok{ }\DecValTok{2}
\NormalTok{    contador =}\StringTok{ }\DecValTok{0}
\NormalTok{    fxInicial =}\StringTok{ }\KeywordTok{Fx}\NormalTok{(a)}
\NormalTok{    done =}\StringTok{ }\OtherTok{FALSE}
    
    \ControlFlowTok{while}\NormalTok{(}\KeywordTok{abs}\NormalTok{(xInicial }\OperatorTok{-}\StringTok{ }\NormalTok{fxInicial) }\OperatorTok{>}\StringTok{ }\NormalTok{error)\{}
\NormalTok{      contador =}\StringTok{ }\NormalTok{contador }\OperatorTok{+}\StringTok{ }\DecValTok{1}
      
      \ControlFlowTok{if}\NormalTok{ (xInicial }\OperatorTok{<}\StringTok{ }\NormalTok{a)\{}
\NormalTok{        done =}\StringTok{ }\OtherTok{TRUE}
\NormalTok{      \}}
      \ControlFlowTok{if}\NormalTok{ (done }\OperatorTok{==}\StringTok{ }\OtherTok{FALSE}\NormalTok{)\{}
\NormalTok{        xInicial =}\StringTok{ }\NormalTok{fxInicial}
\NormalTok{        fxInicial =}\StringTok{ }\KeywordTok{Fx}\NormalTok{(xInicial)}
\NormalTok{      \} }\ControlFlowTok{else}\NormalTok{ \{}
\NormalTok{        fxInicial =}\StringTok{ }\NormalTok{xInicial}
\NormalTok{        xInicial =}\StringTok{ }\KeywordTok{Fx1}\NormalTok{(fxInicial)}
\NormalTok{      \}}
\NormalTok{    \}}
   \KeywordTok{cat}\NormalTok{(}\StringTok{"Iteracciones: "}\NormalTok{, contador, }\StringTok{"Resultado: "}\NormalTok{, xInicial, }\StringTok{"}\CharTok{\textbackslash{}n}\StringTok{"}\NormalTok{) }
\NormalTok{  \}}
\NormalTok{\}}

\KeywordTok{PuntoFijo}\NormalTok{(}\DecValTok{0}\NormalTok{, }\DecValTok{1}\NormalTok{, }\FloatTok{10e-8}\NormalTok{)}
\end{Highlighting}
\end{Shaded}

\includegraphics{Todos_files/figure-latex/unnamed-chunk-3-1.pdf}
\includegraphics{Todos_files/figure-latex/unnamed-chunk-3-2.pdf}

\begin{verbatim}
## El intervalo no es valido
\end{verbatim}

Método de la secante

\begin{Shaded}
\begin{Highlighting}[]
\KeywordTok{rm}\NormalTok{(}\DataTypeTok{list=}\KeywordTok{ls}\NormalTok{())}
\NormalTok{Fx =}\StringTok{ }\ControlFlowTok{function}\NormalTok{(x) }\KeywordTok{exp}\NormalTok{(x) }\OperatorTok{-}\StringTok{ }\NormalTok{pi }\OperatorTok{*}\StringTok{ }\NormalTok{x}
\NormalTok{Fx1 =}\StringTok{ }\ControlFlowTok{function}\NormalTok{(x) }\KeywordTok{exp}\NormalTok{(x) }\OperatorTok{-}\StringTok{ }\NormalTok{pi}

\NormalTok{Secante =}\StringTok{ }\ControlFlowTok{function}\NormalTok{(x1, x2, error)\{}
\NormalTok{  x =}\StringTok{ }\NormalTok{(}\KeywordTok{Fx}\NormalTok{(x2) }\OperatorTok{*}\StringTok{ }\NormalTok{x1 }\OperatorTok{-}\StringTok{ }\KeywordTok{Fx}\NormalTok{(x1) }\OperatorTok{*}\StringTok{ }\NormalTok{x2) }\OperatorTok{/}\StringTok{ }\NormalTok{(}\KeywordTok{Fx}\NormalTok{(x2) }\OperatorTok{-}\StringTok{ }\KeywordTok{Fx}\NormalTok{(x1))}
\NormalTok{  err =}\StringTok{ }\DecValTok{1}
\NormalTok{  contador =}\StringTok{ }\DecValTok{0}
  \ControlFlowTok{while}\NormalTok{ (err }\OperatorTok{>}\StringTok{ }\NormalTok{error)\{}
\NormalTok{    contador =}\StringTok{ }\NormalTok{contador }\OperatorTok{+}\StringTok{ }\DecValTok{1}
\NormalTok{    x1 =}\StringTok{ }\NormalTok{x2}
\NormalTok{    x2 =}\StringTok{ }\NormalTok{x}
\NormalTok{    x =}\StringTok{ }\NormalTok{(}\KeywordTok{Fx}\NormalTok{(x2) }\OperatorTok{*}\StringTok{ }\NormalTok{x1 }\OperatorTok{-}\StringTok{ }\KeywordTok{Fx}\NormalTok{(x1) }\OperatorTok{*}\StringTok{ }\NormalTok{x2) }\OperatorTok{/}\StringTok{ }\NormalTok{(}\KeywordTok{Fx}\NormalTok{(x2) }\OperatorTok{-}\StringTok{ }\KeywordTok{Fx}\NormalTok{(x1))}
    \ControlFlowTok{if}\NormalTok{ (}\KeywordTok{Fx}\NormalTok{(x) }\OperatorTok{==}\StringTok{ }\DecValTok{0}\NormalTok{)}
      \ControlFlowTok{break}
\NormalTok{    err =}\StringTok{ }\KeywordTok{abs}\NormalTok{(}\KeywordTok{Fx}\NormalTok{(x) }\OperatorTok{/}\StringTok{ }\KeywordTok{Fx1}\NormalTok{(x))}
    \KeywordTok{cat}\NormalTok{(}\StringTok{"Valor X: "}\NormalTok{, x, }\StringTok{"}\CharTok{\textbackslash{}t\textbackslash{}t}\StringTok{Valor del Error: "}\NormalTok{, err, }\StringTok{"}\CharTok{\textbackslash{}t\textbackslash{}t}\StringTok{Iteraccion: "}\NormalTok{, contador, }\StringTok{"}\CharTok{\textbackslash{}n}\StringTok{"}\NormalTok{)}
\NormalTok{  \}}
\NormalTok{\}}

\KeywordTok{Secante}\NormalTok{(}\DecValTok{0}\NormalTok{, }\DecValTok{1}\NormalTok{, }\FloatTok{10e-8}\NormalTok{)}
\end{Highlighting}
\end{Shaded}

\begin{verbatim}
## Valor X:  0.464349       Valor del Error:  0.08524539        Iteraccion:  1 
## Valor X:  0.5626182      Valor del Error:  0.008839985       Iteraccion:  2 
## Valor X:  0.5542804      Valor del Error:  0.0004534648      Iteraccion:  3 
## Valor X:  0.5538245      Valor del Error:  2.495487e-06      Iteraccion:  4 
## Valor X:  0.553827       Valor del Error:  7.02433e-10       Iteraccion:  5
\end{verbatim}

Teorema de Aitken

Método de Aitken

\begin{Shaded}
\begin{Highlighting}[]
\NormalTok{fx<-}\ControlFlowTok{function}\NormalTok{(x)}
\NormalTok{\{}
  \KeywordTok{signif}\NormalTok{(}\KeywordTok{exp}\NormalTok{(}\DecValTok{1}\NormalTok{), }\DecValTok{5}\NormalTok{)}\OperatorTok{^}\NormalTok{x}
\NormalTok{\}}

\NormalTok{fx1<-}\ControlFlowTok{function}\NormalTok{(x)}
\NormalTok{\{}
  \KeywordTok{signif}\NormalTok{(pi,}\DecValTok{5}\NormalTok{)}\OperatorTok{*}\NormalTok{x}
\NormalTok{\}}

\NormalTok{fx2<-}\ControlFlowTok{function}\NormalTok{(x)}
\NormalTok{\{}
  \KeywordTok{signif}\NormalTok{(}\KeywordTok{exp}\NormalTok{(}\DecValTok{1}\NormalTok{), }\DecValTok{5}\NormalTok{)}\OperatorTok{^}\NormalTok{x}\OperatorTok{-}\KeywordTok{signif}\NormalTok{(pi,}\DecValTok{5}\NormalTok{)}\OperatorTok{*}\NormalTok{x}
\NormalTok{\}}

\NormalTok{aitken =}\StringTok{ }\ControlFlowTok{function}\NormalTok{(f, m, x0, tol)}
\NormalTok{\{}
  
  \KeywordTok{plot}\NormalTok{(fx, }\DataTypeTok{xlim =} \KeywordTok{c}\NormalTok{(}\OperatorTok{-}\DecValTok{2}\NormalTok{,}\DecValTok{2}\NormalTok{), }\DataTypeTok{ylim =} \KeywordTok{c}\NormalTok{(}\DecValTok{0}\NormalTok{,}\DecValTok{6}\NormalTok{), }\DataTypeTok{col =} \StringTok{"blue"}\NormalTok{, }\DataTypeTok{main =} \StringTok{"Grafica de las Funciones"}\NormalTok{, }\DataTypeTok{sub =} \StringTok{"Aitken"}\NormalTok{, }\DataTypeTok{xlab =} \StringTok{"x"}\NormalTok{, }\DataTypeTok{ylab =} \StringTok{"y"}\NormalTok{)}
  \KeywordTok{par}\NormalTok{(}\DataTypeTok{new=}\OtherTok{TRUE}\NormalTok{)}
  \KeywordTok{curve}\NormalTok{(fx1, }\DataTypeTok{type =} \StringTok{"l"}\NormalTok{, }\DataTypeTok{col=}\StringTok{"green"}\NormalTok{, }\DataTypeTok{axes=}\OtherTok{FALSE}\NormalTok{, }\DataTypeTok{ylab =} \StringTok{"y"}\NormalTok{)}
  \KeywordTok{par}\NormalTok{(}\DataTypeTok{new=}\OtherTok{FALSE}\NormalTok{)}
  
\NormalTok{  iteraciones<-}\KeywordTok{c}\NormalTok{()}
  
\NormalTok{  Er1<-}\KeywordTok{c}\NormalTok{()}
\NormalTok{  Er2<-}\KeywordTok{c}\NormalTok{()}
  
\NormalTok{  k<-}\DecValTok{0}
\NormalTok{  E1<-}\DecValTok{0}
  
\NormalTok{  g<-}\KeywordTok{parse}\NormalTok{(}\DataTypeTok{text=}\NormalTok{f)}
\NormalTok{  fx =}\StringTok{ }\ControlFlowTok{function}\NormalTok{(x)\{}\KeywordTok{eval}\NormalTok{(g[[}\DecValTok{1}\NormalTok{]])\}}
\NormalTok{  d.<-}\KeywordTok{D}\NormalTok{(}\KeywordTok{parse}\NormalTok{(}\DataTypeTok{text=}\NormalTok{f ), }\StringTok{"x"}\NormalTok{)}
\NormalTok{  df<-}\ControlFlowTok{function}\NormalTok{(x) }\KeywordTok{eval}\NormalTok{(d.)}
  
  \KeywordTok{plot}\NormalTok{(fx, }\DataTypeTok{xlim =} \KeywordTok{c}\NormalTok{(}\OperatorTok{-}\FloatTok{0.5}\NormalTok{,}\DecValTok{5}\NormalTok{), }\DataTypeTok{ylim =} \KeywordTok{c}\NormalTok{(}\OperatorTok{-}\DecValTok{2}\NormalTok{,}\DecValTok{5}\NormalTok{), }\DataTypeTok{col =} \StringTok{"blue"}\NormalTok{, }\DataTypeTok{main =} \StringTok{"Grafica funcion"}\NormalTok{, }\DataTypeTok{sub =} \StringTok{"Aitken"}\NormalTok{, }\DataTypeTok{xlab =} \StringTok{"x"}\NormalTok{, }\DataTypeTok{ylab =} \StringTok{"y"}\NormalTok{)}
  \KeywordTok{abline}\NormalTok{(}\DataTypeTok{h =} \DecValTok{0}\NormalTok{, }\DataTypeTok{v=}\DecValTok{0}\NormalTok{, }\DataTypeTok{col=} \StringTok{"red"}\NormalTok{)}
  
  \ControlFlowTok{repeat}
\NormalTok{  \{}
    
\NormalTok{    x1 =}\StringTok{ }\NormalTok{x0 }\OperatorTok{-}\StringTok{ }\NormalTok{m}\OperatorTok{*}\NormalTok{(}\KeywordTok{fx}\NormalTok{(x0)}\OperatorTok{/}\KeywordTok{df}\NormalTok{(x0))}
\NormalTok{    dx =}\StringTok{ }\KeywordTok{abs}\NormalTok{(x1}\OperatorTok{-}\NormalTok{x0)}
\NormalTok{    E2 =}\StringTok{ }\NormalTok{E1}
\NormalTok{    E1 =}\StringTok{ }\NormalTok{dx}\OperatorTok{/}\NormalTok{x1}
    \KeywordTok{cat}\NormalTok{(}\StringTok{"X="}\NormalTok{, x1, }\StringTok{"}\CharTok{\textbackslash{}t}\StringTok{"}\NormalTok{, }\StringTok{"E="}\NormalTok{, dx, }\StringTok{"}\CharTok{\textbackslash{}t}\StringTok{ e="}\NormalTok{, E1,}\StringTok{"}\CharTok{\textbackslash{}t}\StringTok{ Iteracion"}\NormalTok{, k}\OperatorTok{+}\DecValTok{1}\NormalTok{,}\StringTok{"}\CharTok{\textbackslash{}n}\StringTok{"}\NormalTok{)}
    
    \ControlFlowTok{if}\NormalTok{(k }\OperatorTok{>=}\StringTok{ }\DecValTok{1}\NormalTok{)}
\NormalTok{    \{}
\NormalTok{      Er1<-}\KeywordTok{c}\NormalTok{(Er1, E2)}
\NormalTok{      Er2<-}\KeywordTok{c}\NormalTok{(Er2, E1)}
\NormalTok{    \}}
    
\NormalTok{    k =}\StringTok{ }\NormalTok{k }\OperatorTok{+}\StringTok{ }\DecValTok{1}
    
    \ControlFlowTok{if}\NormalTok{ (dx }\OperatorTok{<}\StringTok{ }\NormalTok{tol) }\ControlFlowTok{break}\NormalTok{;}
    
\NormalTok{    x0 =}\StringTok{ }\NormalTok{x1}
    
    
\NormalTok{  \}}
  
  \KeywordTok{points}\NormalTok{(x1,}\DecValTok{0}\NormalTok{)}
  
  \KeywordTok{plot}\NormalTok{(fx, }\DataTypeTok{xlim =} \KeywordTok{c}\NormalTok{(}\DecValTok{0}\NormalTok{,}\KeywordTok{max}\NormalTok{(Er1)), }\DataTypeTok{ylim =} \KeywordTok{c}\NormalTok{(}\DecValTok{0}\NormalTok{,}\KeywordTok{max}\NormalTok{(Er2)), }\DataTypeTok{col =} \StringTok{"white"}\NormalTok{, }\DataTypeTok{main =} \StringTok{"Errores(i) vs Errores(i+1)"}\NormalTok{, }\DataTypeTok{sub =} \StringTok{"Aitken"}\NormalTok{, }\DataTypeTok{xlab =} \StringTok{"Errores(i)"}\NormalTok{, }\DataTypeTok{ylab =} \StringTok{"Errores(i+1)"}\NormalTok{)}
  \KeywordTok{lines}\NormalTok{(Er1, Er2, }\DataTypeTok{type =} \StringTok{"l"}\NormalTok{)}
  
\NormalTok{  Er1<-}\KeywordTok{c}\NormalTok{(Er1,Er2[k])}
\NormalTok{  iteraciones<-}\KeywordTok{c}\NormalTok{(}\DecValTok{1}\OperatorTok{:}\NormalTok{k)}
  \KeywordTok{plot}\NormalTok{(fx, }\DataTypeTok{xlim =} \KeywordTok{c}\NormalTok{(}\DecValTok{0}\NormalTok{,iteraciones[k]), }\DataTypeTok{ylim =} \KeywordTok{c}\NormalTok{(}\DecValTok{0}\NormalTok{,Er1[}\DecValTok{1}\NormalTok{]), }\DataTypeTok{col =} \StringTok{"white"}\NormalTok{, }\DataTypeTok{main =} \StringTok{"Iteraciones vs Errores"}\NormalTok{, }\DataTypeTok{sub =} \StringTok{"Aitken"}\NormalTok{, }\DataTypeTok{xlab =} \StringTok{"Iteraciones"}\NormalTok{, }\DataTypeTok{ylab =} \StringTok{"Errores"}\NormalTok{)}
  \KeywordTok{lines}\NormalTok{(iteraciones, Er1, }\DataTypeTok{type =} \StringTok{"l"}\NormalTok{)}
\NormalTok{\}}

\KeywordTok{aitken}\NormalTok{(}\StringTok{"2.7182^x-3.1415*x"}\NormalTok{, }\DecValTok{1}\NormalTok{, }\DecValTok{2}\NormalTok{, }\DecValTok{10}\OperatorTok{^-}\DecValTok{8}\NormalTok{)}
\end{Highlighting}
\end{Shaded}

\includegraphics{Todos_files/figure-latex/unnamed-chunk-6-1.pdf}
\includegraphics{Todos_files/figure-latex/unnamed-chunk-6-2.pdf}

\begin{verbatim}
## X= 1.739666   E= 0.2603344    e= 0.1496462    Iteracion 1 
## X= 1.6496     E= 0.09006603   e= 0.05459873   Iteracion 2 
## X= 1.638732   E= 0.01086763   e= 0.00663173   Iteracion 3 
## X= 1.638579   E= 0.0001525962     e= 9.312712e-05     Iteracion 4 
## X= 1.638579   E= 2.987868e-08     e= 1.823451e-08     Iteracion 5 
## X= 1.638579   E= 1.776357e-15     e= 1.084084e-15     Iteracion 6
\end{verbatim}

\includegraphics{Todos_files/figure-latex/unnamed-chunk-6-3.pdf}
\includegraphics{Todos_files/figure-latex/unnamed-chunk-6-4.pdf}

\begin{Shaded}
\begin{Highlighting}[]
\KeywordTok{aitken}\NormalTok{(}\StringTok{"2.7182^x-3.1415*x"}\NormalTok{, }\DecValTok{1}\NormalTok{, }\DecValTok{0}\NormalTok{, }\DecValTok{10}\OperatorTok{^-}\DecValTok{8}\NormalTok{)}
\end{Highlighting}
\end{Shaded}

\includegraphics{Todos_files/figure-latex/unnamed-chunk-6-5.pdf}
\includegraphics{Todos_files/figure-latex/unnamed-chunk-6-6.pdf}

\begin{verbatim}
## X= 0.4669558      E= 0.4669558    e= 1    Iteracion 1 
## X= 0.5498345      E= 0.08287861   e= 0.1507338    Iteracion 2 
## X= 0.5538331      E= 0.003998596      e= 0.007219858      Iteracion 3 
## X= 0.553843   E= 9.896336e-06     e= 1.786849e-05     Iteracion 4 
## X= 0.553843   E= 6.078171e-11     e= 1.097454e-10     Iteracion 5
\end{verbatim}

\includegraphics{Todos_files/figure-latex/unnamed-chunk-6-7.pdf}
\includegraphics{Todos_files/figure-latex/unnamed-chunk-6-8.pdf}


\end{document}
